\chapter{Opis projektnog zadatka}

		Gubitak kućnog ljubimca može biti jedna od emocionalno najtežih stvari kroz koje vlasnik može proći. Ponekad se dogodi da ljubimac odluta od kuće zbog znatiželje, iznenadnog događaja koji je u njima probudio strah... U tom slučaju, vlasniku je prioritet brzo pronaći mezimca kako bi bio na sigurnom u svom domu.
		
		Naša aplikacija je namijenjena onima koji su izgubili kućnog ljubimca, onima koji žele pomoći drugima u pronalasku svojih krznenih prijatelja pa i skloništima koji pod svoje primaju odlutale prestrašene životinje. Svi zainteresirani za dobrobit ovih ljubimaca imaju direktan pristup svim informacijama o njima. Značaj aplikacije za zajednicu je da može doprinijeti smanjenju broja napuštenih ljubimaca i time olakšati rad skloništima za životinje te promovirati svijest o izgubljenim ljubimcima. \\
		
		Ova korisna i jednostavna responzivna aplikacija pomoći će u rješavanju ovog problema mnogim korisnicima aplikacije.\\
		
		Cilj ovog projekta je razviti programsku potporu za stvaranje web aplikacije „Nestali ljubimci“. U opseg projektnog zadatka ulazi izrada web platforme koja podržava registraciju korisnika/skloništa, postavljanje, pretragu i ažuriranje postojećih oglasa te ima pridruženu bazu podataka koja pohranjuje korisne informacije. Sustav treba podržavati rad više korisnika u stvarnom vremenu. Manipuliranje podacima obavlja se kroz sučelje baze podataka tako da nije potreban administrator.
		
		\eject
		
		\section{Postojeća slična rješenja}
		
		Model našeg projekta moći će se koristiti na globalnoj razini zbog svoje jednostavnosti i prilagodljivosti lokaciji. Već postoji nekoliko web stranica s kojima dijelimo zajednički cilj poput \textit{PetFinder}, \textit{LostMyDoggie.com}, \textit{PawBoost} i \textit{Petco Love}.
		
		\begin{packed_item}		
		
			\item \textit{PetFinder} je naširoko poznata baza podataka za udomljavanje životinja, a imaju i odjeljak za tražene i pronađene kućne ljubimce.
		
			\item \textit{LostMyDoggie.com} je web stranica koja je napravljena specijalno kako bi pomogla ožalošćenim vlasnicima pronaći svoje kućne ljubimce.
		
			\item \textit{PawBoost} je platforma koja omogućuje korisnicima da prijave nestanak ljubimca nakon čega stranica stvori oglas na Facebooku čime se širi vijest o nestanku Vašeg ljubimca.
		
			\item \textit{Petco Love} je web stranica na kojoj se može prijaviti nestanak, ali i pronalazak kućnog ljubimca. Pri tome se šalju i slike ljubimca te se koristi \textit{facial recognition technology} za identificiranje ljubimaca.
		
		\end{packed_item}
		
		U našoj regiji se ipak nešto više komunicira preko raznih Facebook grupa, lokalnih skloništa za životinje, oglašavanja preko veterinara i lijepljenjem papira s oglasom po ulicama.\\
		
		%unos slike
		\begin{figure}[H]
			\includegraphics[scale=0.5]{slike/pawBoost.PNG} 
			\centering
			\caption{oglas platforme PawBoost}
			\label{pawBoost}
		\end{figure}
		
		\begin{figure}[H]
			\includegraphics[scale = 0.4]{slike/petcoLove.PNG} 
			\centering
			\caption{usluge koje nudi platforma PetcoLove}
			\label{petcoLove}
		\end{figure}
		
		\begin{figure}[H]
			\includegraphics[scale = 0.5]{slike/petFinder.PNG} 
			\centering
			\caption{web stranica platforme PetFinder}
			\label{petcoLove}
		\end{figure}
		
		\pagebreak
		
		
		\section{Moguće prilagodbe i nadogradnje rješenja}
		
		\begin{packed_item}
		
			\item Lokaliziranjem aplikacije ona bi postala dostupna i korisnicima u zemljama s drugim jezicima, zakonima i običajima.
			
			\item Osnovni princip naše aplikacije se može primijeniti i na razne izgubljene predmete. Naša aplikacija je svojevrsni \textit{lost and found} (rekonstrukcija nestanka) primijenjen na ljubimce.
			
			\item Nakon početne implementacije, neke od mogućih projektnih nadogradnji uključuju i \textit{real time chat} opciju. Korisnici mogu međusobno privatno komunicirati i dijeliti razne informacije o nestalim ljubimcima. Registrirani korisnik s informacijama koje mogu pomoći vlasniku izgubljenog ljubimca može kontaktirati dotičnog vlasnika koji je objavio oglas.
			
			\item Uvođenje naprednih algoritama za prepoznavanje životinja putem fotografija olakšalo bi i ubrzalo pronalazak.
			
			\item Povezivanjem podataka veterinarskih klinika i baze prijavljenih životinja aplikacije može doći do preklapanja upita u bazi podataka. Time bi se napravio korak dalje u zbližavanju vlasnika i nestalog ljubimca.
			
			\item Na web aplikaciji bi mogla postojati mogućnost donacije sredstava lokalnim skloništima za životinje.\\
			
		\end{packed_item}
		
		Web aplikacija je namijenjena za 3 vrste korisnika; neregistriranom korisniku, registriranom te skloništima za životinje (specijalni tip registriranog korisnika).\\
		
		\underline{Neregistrirani korisnik} ima mogućnost pregledavanja i pretraživanja nestalih kućnih ljubimaca. Klikom na sliku nestale životinje, znatiželjnom korisniku otvaraju se informacije o njoj: vrsta, ime, datum i sat nestanka, lokacija nestanka, boja, starost, tekstni opis. Uz to dostupne su do 3 slike te životinje kako bi ju tragač lakše pronašao, a ako bi došlo do novih informacija ili pronalaska ljubimca dostupni su i kontakt podaci vlasnika. Neregistrirani korisnik bi se trebao registrirati ako bi poželio sudjelovati u potrazi i ostvariti komunikaciju s dotičnim vlasnikom. Za registraciju su potrebni sljedeći podaci:
		
		\begin{packed_item}
			\item adresa e-pošte
			\item broj telefona
			\item korisničko ime
			\item lozinka
			\item ime i prezime/naziv skloništa
		\end{packed_item}
		
		\underline{Registrirani korisnik} ima širi spektar mogućnosti unutar aplikacije. On može, uz pregledavanje i pretraživanje, postaviti oglas o nestalom ljubimcu, izmijeniti i ukloniti ga pa i sudjelovati u komunikaciji s drugim registriranim korisnicima.\\
		
		Postoje 4 kategorije oglasa:
		
		\begin{packed_item}
			\item Za ljubimcem se traga (početna postavka oglasa)
			\item Ljubimac je sretno pronađen
			\item Ljubimac nije nađen, ali se za njim ne traga aktivno
			\item Ljubimac je pronađen pod nesretnim okolnostima
		\end{packed_item}
		
		\underline{Skloništa za životinje} su vrsta registriranih korisnika koji oglašavaju životinje koje su pronašli te se nalaze kod njih.\\\\
		
		\begin{longtblr}[
			label=none,
			entry=none
			]{
				width = \textwidth,
				colspec={|X[8,l]|X[8, l]|X[16, l]|}, 
				rowhead = 1,
			} %definicija širine tablice, širine stupaca, poravnanje i broja redaka naslova tablice
			\hline \SetCell[c=3]{c}{\textbf{naslov unutar tablice}}	 \\ \hline[3pt]
			\SetCell{LightGreen}IDKorisnik & INT	&  	Lorem ipsum dolor sit amet, consectetur adipiscing elit, sed do eiusmod  	\\ \hline
			korisnickoIme	& VARCHAR &   	\\ \hline 
			email & VARCHAR &   \\ \hline 
			ime & VARCHAR	&  		\\ \hline 
			\SetCell{LightBlue} primjer	& VARCHAR &   	\\ \hline 
		\end{longtblr}
		

		\begin{longtblr}[
				caption = {Naslov s referencom izvan tablice},
				entry = {Short Caption},
			]{
				width = \textwidth, 
				colspec = {|X[8,l]|X[8,l]|X[16,l]|}, 
				rowhead = 1,
			}
			\hline
			\SetCell{LightGreen}IDKorisnik & INT	&  	Lorem ipsum dolor sit amet, consectetur adipiscing elit, sed do eiusmod  	\\ \hline
			korisnickoIme	& VARCHAR &   	\\ \hline 
			email & VARCHAR &   \\ \hline 
			ime & VARCHAR	&  		\\ \hline 
			\SetCell{LightBlue} primjer	& VARCHAR &   	\\ \hline 
		\end{longtblr}
		
		\eject
		
	