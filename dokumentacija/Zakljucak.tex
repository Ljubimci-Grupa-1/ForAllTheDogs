\chapter{Zaključak i budući rad}
		
		Zadatak naše grupe bio je napraviti web aplikaciju koja bi pomogla opet ujediniti vlasnike sa svojim nestalim ljubimcima. Aplikacija služi kao svojevrsni portal s oglasima o izgubljenim, ali i pronađenim životinjama.\\
		Aplikaciju smo izrađivali tijekom oba nastavna ciklusa predmeta što je trajalo otprilike 14 tjedana. U svakom ciklusu smo imali zadatke koje smo morali ispuniti do unaprijed zadanog roka, a na kraju svakog ciklusa se ocjenjivao naš rad i uspjeh projektnog zadatka.\\
		U prvoj reviziji projekta, naglasak je bio na dokumentaciji i na izradi temelja za programsku potporu naše web aplikacije. Na početku projekta znali smo da je važno imati dobru povezanost i komunikaciju unutar tima te je uslijedilo upoznavanje članova, podjela posla i rasprava o potencijalnim načinima i tehnologijama pomoću kojih bismo mogli provesti ovaj projekt. Posao smo podijelili po parovima za izradu klijentske i poslužiteljske strane aplikacije te izradu dokumentacije, a znalo se dogoditi i da si svi članovi međusobno pomažu čime smo ubrzali proces izrade aplikacije, ali i podijelili znanje. Krenuli smo izradom modela baze podataka i pisanjem dokumentacije, a kako smo se bližili kraju prve faze projekta krenuli smo i s implementacijom pojedinih obrazaca uporabe i povezivanjem klijentske i poslužiteljske strane. Ipak, izrada dokumentacije je imala glavnu riječ u ovoj fazi projekta. Shvatili smo da trebamo velik fokus staviti na dijagram baze podataka, smišljanje obrazaca uporabe, sekvencijskih dijagrama i dijagram razreda jer će upravo oni biti podloga za razvoj programske potpore naše aplikacije. Pomogli su nam da unaprijed uvidimo gdje bi mogli izviriti problemi i koji su dijelovi ključni za dobru izradu i izvedbu našeg portala.\\
		Za drugu reviziju naglasak se prebacio na samu implementaciju programske potpore za sve funkcionalnosti koje nisu uključene u izvedbu tijekom prve revizije. U ovom dijelu su članovi tima koji su radili \textit{back-end} puno više komunicirali s onima koji su radili \textit{front-end} jer su njihovi zadaci ovisili jedni od drugima. Dijagrami su u ovoj fazi prikazivali stanja objekata i njihove prijelaze, način funkcioniranja aplikacije (odnos između korisnika, web stranice i baze podataka) te organizaciju i međuovisnost komponenti.\\
		Očekivali smo da ćemo projekt uspješno svladati, iako su svi članovi imali različita iskustva i znanja. To se i dogodilo, a svaki je član ponešto novo naučio i upoznao se s tehnologijom koju je koristio pri rješavanju svog dijela projektnog zadatka. Naravno, svi su članovi međusobno dijelili sve što su do nekog trenutka naučili. Na početku su možda iskusniji članovi uzeli inicijativu u svoje ruke, ali s rastom projekta i novim naučenim znanjem drugi su ih članovi dostigli i svi su jednako doprinosili projektu. Članovi koji su se već bili susreli s potencijalnim problemima znali su kako reagirati i po mogućnosti ih izbjeći. Kada bi jedan član završio s nekim dijelom zadatka, drugi član/ovi bi pregledali zadovoljava li napravljeno sve što je i trebalo i time bismo skratili vrijeme izrade projekta jer bi moguća greška bila na vrijeme uočena.\\
		Komunikacija vezana uz projekt održana je preko Discord-a gdje su postojali kanali za različite dijelove projekta. Najčešće smo komunicirali preko poruka, a sastanci su se odvijali uživo i preko video-poziva. \\
		Sudjelovanjem u ovom projektu svi su članovi nešto novo naučili, čak i oni koji su imali nešto više znanja od drugih. Možemo reći da se svi ponešto odvažnije osjećamo po pitanju ovakvog tipa projektnog zadatka. Na kraju projekta smo shvatili što se sve od nas očekuje pri izradi web aplikacije te kako sve zahtjeve implementirati i povezati tako da sustav bude samoodrživ. Također smo shvatili da svaki član mora dati sve od sebe za dio zadatka kojeg radi, proučiti literaturu i istražiti nešto novo kako bismo probali izbjeći zaostatke i imali kontinuirani rast i razvoj web aplikacije.
		
		\eject 