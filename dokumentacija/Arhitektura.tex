\chapter{Arhitektura i dizajn sustava}
		
		Arhitektura se može podijeliti na 4 podsustava:
		
		\begin{packed_item}
		
			\item 	Web poslužitelj
			\item 	Front-end
			\item 	Back-end		
			\item   Baza podataka
		\end{packed_item}
		
		\textit{Web preglednik} je program koji omogućuje korisnicima pristupanje internet resursima putem zahtjeva poslanih poslužiteljima. Djeluje kao klijent koji traži te resurse. Dohvaćanje tih resursa odvija se putem API poziva koje korisnik obavlja preko \textit{front-enda}.
		
		\textit{Front-end} web aplikacija omogućuje korisniku interakciju s \textit{back-end}-om na poslužitelju, koje se omogućuje kroz korisničko sučelje. 
		\textit{Front-end} aplikacija šalje zahtjeve \textit{back-end} aplikaciji, koja ih prima, obradi, te po potrebi pristupa bazi podataka, šaljući potvrdne ili neuspješne odgovore natrag \textit{front-end} aplikaciji, uključujući dodatne podatke u tijelu odgovora kad je potrebno.\\
		
		\section{Programski jezici, razvojni okvir, alati i biblioteke koda}
		
		
		\subsection{Back-end i baza podataka}
		U okviru \textit{back-end} aplikacije koriste se različiti alati i tehnologije kako bi se postigla funkcionalnost web aplikacije. 
		
		Sama funkcionalnost \textit{back-enda} ostvarena je koristeći \textit{Kotlin} i \textit{Spring Boot}, popularni \textit{framework} za Javu i Kotlin.
		
		\textit{Spring Boot} olakšava izradu web aplikacije pružajući razne automatske konfiguracije. To uvelike omogućava integraciju različitih dijelova aplikacije i pruža mnogo gotovih implementacija koje se koriste putem vrlo intuitivnih sučelja.
		
		Baza podataka ostvarena je u \textit{PostgreSQL}-u, a za njenu jasnu definiciju korišten je alat \textit{Liquibase}. \textit{Liquibase} omogućava precizno i upravljivo definiranje strukture baze podataka.
		
		Za preslikavanje entiteta iz \textit{PostgreSQL} baze podataka na klase u backend aplikaciji korišten je \textit{JPA (Jakarta Persistence API)}, koji značajno olakšava generiranje upita ovisno o pozivima metoda nad klasama entiteta. 
		
		Za konfiguracijske datoteke aplikacije koristi se \textit{YAML} format. Upravljanje bazom podataka odvija se preko \textit{H2} konzole, a razvoj kompletne backend aplikacije obavlja se u \textit{IntelliJ IDEA}, popularnom alatu kompanije \textit{JetBrains}. 
		
		Za izgradnju cijele aplikacije koristi se \textit{Gradle}, alat za automatizaciju izgradnje. 
\textit{JWT (Json Web Token}) standard korišten je za sigurnost aplikacije, generirajući tokene, koji istovremeno služe za autorizaciju i autentifikaciju korisnika.


			\subsection{Front-end}
			U izradi \textit{front-end} dijela aplikacije koristimo niz tehnologija kako bismo postigli željene funkcionalnosti i estetski privlačan dizajn. Ključne tehnologije koje se koriste u razvoju uključuju \textit{TypeScript}, \textit{React}, \textit{Bootstrap}, \textit{Vite} i \textit{IntelliJ}.
			
\textit{React}, kao glavni okvir, omogućava olakšanu izradu web stranica i pruža širok spektar alata za navigaciju, dohvaćanje i prikazivanje podataka. Koristi se označni kod sličan \textit{HTML}-u, obogaćen mogućnostima \textit{TypeScript}-a za definiranje sadržaja stranica, dok se za definiranje stila i izgleda koristi \textit{Bootstrap}.

Za efikasno upravljanje podacima u aplikaciji koristi se \textit{React Query}. \textit{Vite}, alat za brzu izgradnju aplikacija, osigurava optimiziran razvojni proces i ubrzanje vremena učitavanja stranica.

Konačno, za pisanje \textit{TypeScript} koda koristi se \textit{IntelliJ}, moćno razvojno okruženje koje omogućava precizno kodiranje i upravljanje projektom. Ovaj skup tehnologija omogućava nam izradu kvalitetne \textit{front-end} aplikacije s visokom funkcionalnošću i atraktivnim dizajnom.\\
		
		
				
		\section{Baza podataka}
			
		

			\subsection{Opis tablica}
			

				\textbf{APP\_USER} Predstavlja registriranog korisnika aplikacije, u \textit{@ManyToOne} vezi s \textbf{USER\_TYPE} preko \textit{userTypeId}.
				
				
				\begin{longtblr}[
					label=none,
					entry=none
					]{
						width = \textwidth,
						colspec={|X[8,l]|X[6, l]|X[18, l]|}, 
						rowhead = 1,
					} %definicija širine tablice, širine stupaca, poravnanje i broja redaka naslova tablice
					\hline \SetCell[c=3]{c}{\textbf{APP\_USER}}	 \\ \hline[3pt]
					\SetCell{LightGreen}userId & BIGINT	&  	jedinstveni ID korisnika  	\\ \hline
					username	& VARCHAR &   korisničko ime (jedinstveno)	\\ \hline 
					email & VARCHAR &   korisnikova e-mail adresa (jedinstvena)	\\ \hline 
					password & VARCHAR	&  	korisnikova lozinka	\\ \hline 
					name & VARCHAR	&  	ime korisnika	\\ \hline 
					telephone\_number & VARCHAR	&  	korisnikov broj telefona (jedinstven)	\\ \hline 
					\SetCell{LightBlue} userTypeId	& BIGINT &   ID koji označava tip korisnika	\\ \hline 
				\end{longtblr}
				
				\noindent\textbf{USER\_TYPE} Predstavlja popis tipova korisnika aplikacije (userTypeId=1 odgovara osobi, userTypeId=2 skloništu).
				
				
				\begin{longtblr}[
					label=none,
					entry=none
					]{
						width = \textwidth,
						colspec={|X[8,l]|X[6, l]|X[18, l]|}, 
						rowhead = 1,
					} %definicija širine tablice, širine stupaca, poravnanje i broja redaka naslova tablice
					\hline \SetCell[c=3]{c}{\textbf{USER\_TYPE}}	 \\ \hline[3pt]
					\SetCell{LightGreen}userTypeId & BIGINT	&  	jedinstveni ID tipa korisnika  	\\ \hline
					name	& VARCHAR &   ime tipa korisnika	\\ \hline 
				\end{longtblr}
				
				\noindent\textbf{MESSAGE} Poruka koju korisnici mogu ostavljati u komunikaciji ispod oglasa, u \textit{@ManyToOne} vezi s \textbf{AD} preko \textit{adId}, \textit{@ManyToOne} vezi s \textbf{USER} preko \textit{userId}, \textit{@OneToOne} vezi s \textbf{IMAGE} preko \textit{imageId}.
				
				\begin{longtblr}[
					label=none,
					entry=none
					]{
						width = \textwidth,
						colspec={|X[8,l]|X[6, l]|X[18, l]|}, 
						rowhead = 1,
					} %definicija širine tablice, širine stupaca, poravnanje i broja redaka naslova tablice
					\hline \SetCell[c=3]{c}{\textbf{MESSAGE}}	 \\ \hline[3pt]
					\SetCell{LightGreen}messageId & BIGINT	&  	jedinstveni ID poruke  	\\ \hline
					text	& VARCHAR &   sadržaj poruke	\\ \hline 
					date	& DATE &   datum kada je poruka ostavljena ispod oglasa	\\ \hline 
					location	& VARCHAR &   lokacija s koje je poruka ostavljena	\\ \hline 
					\SetCell{LightBlue}adId	& BIGINT &   jedinstveni ID oglasa ispod kojeg je poruka ostavljena	\\ \hline 
					\SetCell{LightBlue}userId	& BIGINT &   jedinstveni ID korisnika koji je ostavio poruku	\\ \hline 
					\SetCell{LightBlue}imageId	& BIGINT &   jedinstveni ID slike ostavljene uz poruku	\\ \hline 
				\end{longtblr}
				
				\noindent\textbf{IMAGE} Tablica u koju se spremaju slike koje se dohvaćaju preko URL.
				
				\begin{longtblr}[
					label=none,
					entry=none
					]{
						width = \textwidth,
						colspec={|X[8,l]|X[6, l]|X[18, l]|}, 
						rowhead = 1,
					} %definicija širine tablice, širine stupaca, poravnanje i broja redaka naslova tablice
					\hline \SetCell[c=3]{c}{\textbf{IMAGE}}	 \\ \hline[3pt]
					\SetCell{LightGreen}imageId & BIGINT	&  	jedinstveni ID slike  	\\ \hline
					imageUrl	& VARCHAR &   URL slike	\\ \hline 
				\end{longtblr}
				
				\noindent\textbf{ACTIVITY} Predstavlja kategoriju oglasa.
				
				\begin{longtblr}[
					label=none,
					entry=none
					]{
						width = \textwidth,
						colspec={|X[8,l]|X[6, l]|X[18, l]|}, 
						rowhead = 1,
					} %definicija širine tablice, širine stupaca, poravnanje i broja redaka naslova tablice
					\hline \SetCell[c=3]{c}{\textbf{ACTIVITY}}	 \\ \hline[3pt]
					\SetCell{LightGreen}activityId & BIGINT	&  	jedinstveni ID kategorije  	\\ \hline
					activityCategory	& VARCHAR &   naziv kategorije	\\ \hline 
				\end{longtblr}
				
				\noindent\textbf{AD} Predstavlja oglas, u \textit{@ManyToOne} vezi s \textbf{ACTIVITY}, \textit{@ManyToOne} vezi s \textbf{USER} preko \textit{userId}, \textit{@OneToOne} vezi s \textbf{IMAGE} (moguće 3 slike po oglasu), \textit{@OneToOne} vezi s \textbf{PET}.
				
				\begin{longtblr}[
					label=none,
					entry=none
					]{
						width = \textwidth,
						colspec={|X[8,l]|X[6, l]|X[18, l]|}, 
						rowhead = 1,
					} %definicija širine tablice, širine stupaca, poravnanje i broja redaka naslova tablice
					\hline \SetCell[c=3]{c}{\textbf{AD}}	 \\ \hline[3pt]
					\SetCell{LightGreen}adId & BIGINT	&  	jedinstveni ID oglasa  	\\ \hline
					inShelter	& INT &   1 ako je oglas od skloništa, inače 0 	\\ \hline 
					\SetCell{LightBlue}userId	& BIGINT &   jedinstveni ID korisnika koji je postavio oglas 	\\ \hline 
					\SetCell{LightBlue}activityId	& BIGINT &   jedinstveni ID kategorije oglasa 	\\ \hline
					\SetCell{LightBlue}image1Id	& BIGINT &   jedinstveni ID prve slike 	\\ \hline
					\SetCell{LightBlue}image2Id	& BIGINT &   jedinstveni ID druge slike 	\\ \hline
					\SetCell{LightBlue}image3Id	& BIGINT &   jedinstveni ID treće slike 	\\ \hline
					\SetCell{LightBlue}petId	& BIGINT &   jedinstveni ID ljubimca u oglasu 	\\ \hline
				\end{longtblr}
				
				\noindent\textbf{PET} Predstavlja ljubimca, u \textit{@ManyToMany} vezi s \textbf{COLOR}.
				
				\begin{longtblr}[
					label=none,
					entry=none
					]{
						width = \textwidth,
						colspec={|X[8,l]|X[6, l]|X[18, l]|}, 
						rowhead = 1,
					} %definicija širine tablice, širine stupaca, poravnanje i broja redaka naslova tablice
					\hline \SetCell[c=3]{c}{\textbf{PET}}	 \\ \hline[3pt]
					\SetCell{LightGreen}petId & BIGINT	&  	jedinstveni ID ljubimca  	\\ \hline
					hourMissing	& INT &   sat nestanka ljubimca	\\ \hline 
					dateMissing	& DATE &   datum nestanka ljubimca	\\ \hline 
					age	& INT &   starost ljubimca	\\ \hline 
					description	& VARCHAR &   opis ljubimca	\\ \hline 
					petName	& VARCHAR &   ime na koje se ljubimac odaziva	\\ \hline 
					species	& VARCHAR &   vrsta ljubimca	\\ \hline 
				\end{longtblr}
				
				\noindent\textbf{COLOR} Tablica s bojama ljubimaca, u \textit{@ManyToMany} vezi s \textbf{PET}.
				
				\begin{longtblr}[
					label=none,
					entry=none
					]{
						width = \textwidth,
						colspec={|X[8,l]|X[6, l]|X[18, l]|}, 
						rowhead = 1,
					} %definicija širine tablice, širine stupaca, poravnanje i broja redaka naslova tablice
					\hline \SetCell[c=3]{c}{\textbf{COLOR}}	 \\ \hline[3pt]
					\SetCell{LightGreen}colorId & BIGINT	&  	jedinstveni ID boje  	\\ \hline
					colorName	& VARCHAR &   naziv boje	\\ \hline 
				\end{longtblr}
				
				\noindent\textbf{OF\_COLOR} Tablica veze između \textbf{PET} i \textbf{COLOR}.
				
				\begin{longtblr}[
					label=none,
					entry=none
					]{
						width = \textwidth,
						colspec={|X[8,l]|X[6, l]|X[18, l]|}, 
						rowhead = 1,
					} %definicija širine tablice, širine stupaca, poravnanje i broja redaka naslova tablice
					\hline \SetCell[c=3]{c}{\textbf{OF\_COLOR}}	 \\ \hline[3pt]
					\SetCell{LightBlue}colorId & BIGINT	&  	jedinstveni ID boje  	\\ \hline
					\SetCell{LightBlue}petId & BIGINT	&  	jedinstveni ID ljubimca
				\end{longtblr}
				
			
			\subsection{Dijagram baze podataka}
				\textit{ U ovom potpoglavlju potrebno je umetnuti dijagram baze podataka. Primarni i strani ključevi moraju biti označeni, a tablice povezane. Bazu podataka je potrebno normalizirati. Podsjetite se kolegija "Baze podataka".}
			
			\eject
			
			
		\section{Dijagram razreda}
		
			\textit{Potrebno je priložiti dijagram razreda s pripadajućim opisom. Zbog preglednosti je moguće dijagram razlomiti na više njih, ali moraju biti grupirani prema sličnim razinama apstrakcije i srodnim funkcionalnostima.}\\
			
			\textbf{\textit{dio 1. revizije}}\\
			
			\textit{Prilikom prve predaje projekta, potrebno je priložiti potpuno razrađen dijagram razreda vezan uz \textbf{generičku funkcionalnost} sustava. Ostale funkcionalnosti trebaju biti idejno razrađene u dijagramu sa sljedećim komponentama: nazivi razreda, nazivi metoda i vrste pristupa metodama (npr. javni, zaštićeni), nazivi atributa razreda, veze i odnosi između razreda.}\\
			
			\textbf{\textit{dio 2. revizije}}\\			
			
			\textit{Prilikom druge predaje projekta dijagram razreda i opisi moraju odgovarati stvarnom stanju implementacije}
			
			
			
			\eject
		
		\section{Dijagram stanja}
			
			
			\textbf{\textit{dio 2. revizije}}\\
			
			\textit{Potrebno je priložiti dijagram stanja i opisati ga. Dovoljan je jedan dijagram stanja koji prikazuje \textbf{značajan dio funkcionalnosti} sustava. Na primjer, stanja korisničkog sučelja i tijek korištenja neke ključne funkcionalnosti jesu značajan dio sustava, a registracija i prijava nisu. }
			
			
			\eject 
		
		\section{Dijagram aktivnosti}
			
			\textbf{\textit{dio 2. revizije}}\\
			
			 \textit{Potrebno je priložiti dijagram aktivnosti s pripadajućim opisom. Dijagram aktivnosti treba prikazivati značajan dio sustava.}
			
			\eject
		\section{Dijagram komponenti}
		
			\textbf{\textit{dio 2. revizije}}\\
		
			 \textit{Potrebno je priložiti dijagram komponenti s pripadajućim opisom. Dijagram komponenti treba prikazivati strukturu cijele aplikacije.}